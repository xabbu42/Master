\usepackage[all]{hypcap}
\usepackage[english]{babel}
\usepackage{amsmath}
\usepackage{amsthm}
\usepackage{verbatim}
\usepackage{bussproofs}
\EnableBpAbbreviations
\nonstopmode

\usepackage{pdflscape}
\usepackage{afterpage}
\usepackage{longtable}
\usepackage{xspace}
\usepackage{fontspec,unicode-math}
\usepackage{newunicodechar}
\usepackage{pbox}
\usepackage{slashed}
\usepackage{fancyhdr}
\pagestyle{fancy}

%necessary for parskip and amsthm to play nice togeter
\begingroup
\makeatletter
   \@for\theoremstyle:=definition,remark,plain\do{%
     \expandafter\g@addto@macro\csname th@\theoremstyle\endcsname{%
        \addtolength\thm@preskip\parskip
     }%
   }
\endgroup

\let\oldproof\proof
\def\proof{\oldproof\unskip}

\newtheorem{theorem}{Theorem}
\newtheorem{lemma}{Lemma}
\newtheorem{corollary}{Corollary}[theorem]
\newtheorem{lcorollary}{Corollary}[lemma]
\theoremstyle{definition}
\newtheorem{definition}{Definition}
\renewcommand\qedsymbol{$\blacksquare$}

\newunicodechar{□}{\ensuremath{\Box}}
\newunicodechar{◇}{\Diamond}
\newunicodechar{↾}{\mathord{\upharpoonright}}
\newunicodechar{⇐}{\ensuremath{\Leftarrow}}
\newunicodechar{⇒}{\ensuremath{\Rightarrow}}
\newunicodechar{▹}{\vartriangleright}
\newunicodechar{˚}{^\circ}
%\newunicodechar{⊀}{\nprec}
\DeclareMathOperator{\Glift}{G3lift}
\DeclareMathOperator{\Glp}{G3lp}
\DeclareMathOperator{\Gs}{G3s}
\DeclareMathOperator{\Cut}{Cut}
\DeclareMathOperator{\sub}{sub}

%for markdown inside of definitions and theorems
\newcommand{\hideFromPandoc}[1]{#1}
\hideFromPandoc{
    \let\Begin\begin
    \let\End\end
}
